\section{Haskell}

\paragraph{Operatoren}
\begin{haskell}
  ==, /=, >, <, <= , >=
\end{haskell}

\paragraph{Arithmetikbefehle}
\begin{haskell}
  2 + 3 = 5
  2 - 3 = -1
  2 * 3 = 6
  3 / 2 = 1.5
  sqrt 9 = 3.0
  max 2 3 = 3
  min 2 3 = 2
  sin x = ...
  div 27 6 = 4
  mod 27 6 = 3
  succ 9 = 10
\end{haskell}

\paragraph{Listenbefehle}
\begin{haskell}
  [1..10] = [1,2,...,10]
  1:[2,3] = [1,2,3]
  [1,2]++[3,4] = [1,2,3,4]
  length [1,2,3] = 3
  head [1,2,3] = 1
  tail [1,2,3] = [2,3], tail [3] = []
  init [1,2,3] = [1,2]
  last [1,2,3] = 3
  null [1,2,3] = False, null [] = True
  take 2 [1,2,3] = [1,2]
  drop 2 [1,2,3] = [3]
  reverse [1,2,3] = [3,2,1]
  maximum [1,3,2] = 3
  minimum [3,2,1] = 1
  sum [1,2,3] = 6
  product [1,2,3] = 6
  2 `elem` [1,2,3] = True 
  map (\x -> x+3) [1,2,3] = [4,5,6]
  map (+3) [1,2,3] = [4,5,6]
  filter (>1) [1,2,3] = [2,3]
  foldr (+) 0 [1,2,3] = (1+(2+(3+0)))
  foldl (+) 0 [1,2,3] = (((0+1)+2)+3)
  concat [[1,2],3] = [1,2,3]
  zipWith (+) [1,2] [3,4] = [4,6]
\end{haskell}

\paragraph{Arithmetikfunktionen}
\begin{haskell}
  -- max
  max x y
    | (x>y) = x
    | otherwise = y

  -- binomial coefficient
  binom n k
    | (k==0 || k==n) = 1
    | otherwise = binom (n-1) (k-1) + binom (n-1) k

  -- factorial default
  fak n = if (n==0) then 1 else (n * fak (n-1))

  -- factorial with accumulator
  fakAcc n acc = if (n==0) then acc else fakAcc (n-1) (n * acc)
  fak n = fakAcc n 1

  -- fibonacci
  fib n
    | (n==0) = 0
    | (n==1) = 1
    | otherwise = fib (n-1) + fib (n-2)

  -- fibonacci with accumulators
  fibAcc n n1 n2
    | (n==0) = n1
    | (n==1) = n2
    | otherwise fibAcc (n-1) n2 (n1+n2)
\end{haskell}

\paragraph{Listenfunktionen}
\begin{haskell}
  -- list length (not needed, see list above)
  length l = if (null l) then 0 else 1+(length(tail l))

  -- list length using pattern matching
  length [] = 0
  length (x:xs) = 1 + length xs

  -- list maximum (not needed, see list above)
  lmax [] = error "empty"
  lmax(x:[]) = x
  lmax (x:xs) = max x (lmax xs)

  -- append lists (not needed, see list above)
  app [] r = r
  app (x:xs) r = x:(app xs r)

  -- reverse lists
  rev [] = []
  rev (x:xs) = app (rev xs) [x]

  -- prime sieve
  primes :: Integer -> [Integer]
  primes n = sieve [2..n]
    where sieve [] = []
          sieve (p:xs) = p : sieve (filter (not . multipleOf p) xs)
          multipleOf p x = x `mod` p == 0

  -- first n squares
  squares n = [x * x | x <- [0..n]]

  -- quicksort
  qsort (p:ps) =      (qsort (filter (<=p) ps))
                 ++ p:(qsort (filter (>p) ps))
\end{haskell}

\paragraph{Damen-Problem}
\begin{haskell}
  type Conf = [Int]

  successors :: Conf -> [Conf]
  successors board = map (:board) [1..8]
  
  threatens :: Int -> (Int, Int) -> Bool
  threatens row1 (diag, row2) = row1 == row2
    || abs (row1-row2) == diag
  
  legal :: Conf -> Bool
  legal [] = True
  legal (row:rest) = not (any (threatens row) (zip [1..] rest))
  
  solution :: Conf -> Bool
  solution board = (length board) == 8
  
  backtrack :: Conf -> [Conf]
  backtrack conf =
    if (solution conf) then [conf]
    else concat (map backtrack (filter legal (successors conf)))
  
  queensSolutions :: [Conf]
  queensSolutions = backtrack []
\end{haskell}

\paragraph{Klausurlösungen}
\begin{haskell}
  -- recursively sum up list by halfing
  sumDQ2 [] 0 = 0.0
  sumDQ2 [x] 1 = x
  sumDQ2 l n = let n2 = n `div` 2 in
               sumDQ2 (take n2 l) n2 + sumDQ2 (drop n2 l) (n - n2)
  sumDQ xs = sumDQ2 xs (length xs)
\end{haskell}