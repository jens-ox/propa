\section*{Prolog}

\paragraph{Vordefinierte Atome und Prädikate}
\begin{prolog}
  halt % Prolog verlassen
  listing % Inhalt Datenbank ausgeben
  consult('datei') % Datenbank laden
  write(wert) % Wert ausgeben
  format('Hello ~s', [name]) % Ausgabe Hello Name
  nl % Neue Zeile ausgeben
  assert(fakt) % neuen Fakt hinzufügen 
  retract(fakt) % alten Fakt entfernen
\end{prolog}

\paragraph{Arithmetikfunktionen}
\begin{prolog}
  % fibonacci
  fib(0,0).
  fib(1,1).
  fib(X,Y) :- X>1,
    X1 is X-1, X2 is X-2,
    fib(X1,Y1), fib(X2,Y2),
    Y is Y1+Y2.

  % test if natural number.
  nat(0).
  nat(X) :- nat(Y), X is Y+1.

  % compute approximate square root.
  sqrt(X,Y) :- nat(Y),
    Y2 is Y*Y, Y3 is (Y+1)*(Y+1),
    Y2 =< X, X < Y3.

  % test evenness.
  even(0).
  odd(1).
  even(X) :- X > 0, X1 is X-1, odd (X1).
  odd(X) :- X > 1, X1 is X-1, even(X1).
\end{prolog}

\paragraph{Listenfunktionen}
\begin{prolog}
  % tests if list includes element.
  member(X,[X|R]).
  member(X,[Y|R]) :- member(X,R).

  % appends element to list.
  append([],L,L).
  append([X|R],L,[X|T]) :- append(R,L,T).

  % reverses list.
  rev([],[]).
  rev([X|R],Y) :- rev(R,Y1),append(Y1,[X],Y).

  % permutes list
  permute([],[]).
  permute([X|R],P) :-
    permute(R,P1),
    append(A,B,P1),
    append(A,[X|B],P).

  % quicksort
  qsort([],[]).
  qsort([X|R],Y) :- split(X,R,R1,R2),
    qsort(R1,Y1),
    qsort(R2,Y2),
    append(Y1,[X|Y2],Y).

  split(X,[],[],[]).
  split(X,[H|T],[H|R],Y) :- X>H, split(X,T,R,Y).
  split(X,[H|T],R,[H|Y]) :- X=<H, split(X,T,R,Y).
\end{prolog}

\paragraph{Cut-Funktionen}
\begin{prolog}
  % green-cut max
  max(X,Y,X) :- X>Y, !.
  max(X,Y,Y) :- X=<Y.

  % red-cut max
  max(X,Y,Z) :- X>Y, !, Z=X.
  max(X,Y,Y).

  % negation
  not(X) :- call(X), !, fail.
  not(X).
\end{prolog}