\section{Java Parallel}

\paragraph{Functional Interfaces}
\begin{java}
  @FunctionalInterface
  interface Predicate {
    boolean check(int value);
  }

  public int sum(List<Integer> values, Predicate predicate) {
    int result = 0;
    for (int value : values) {
      if (predicate.check(value)) result += value;
    }
    return result;
  }

  // example use
  sum(values, i -> i > 5);
\end{java}

\paragraph{Method References}
\begin{java}
  class SimpleCheckers {
    public static boolean checkGreaterThanFive(Integer value) {
      return value > 5;
    }
  }

  // example use
  sum(values; SimpleCheckers::checkGreaterThanFive);
\end{java}

\paragraph{Thread Class}
\begin{java}
  class SayHelloThread extends Thread {
    public void run () {
      System.out.println("Hello");
    }

    public static void main(String[] args) {
      (new SayHelloThread()).start();
    }
  }
\end{java}
Methods:
\begin{itemize}
  \item \code{run()}: defined thread task
  \item \code{start()}: starts thread
  \item \code{isAlive()}: returns if thread is still running
  \item \code{sleep(long milliseconds)}: sleeps thread for given time
  \item \code{interrupt()}: sets interrupted flag for thread
  \item \code{isInterrupted()}: return whether interrupted flag is set
\end{itemize}

\paragraph{Runnable Class}
\begin{java}
  class SayHelloRunnable implements Runnable {
    public void run() {
      System.out.println("Hello");
    }

    public static void main(String[] args) {
      (new Thread(new SayHelloRunnable())).start();
    }
  }
\end{java}

\paragraph{Runnable using Lambda}
\begin{java}
    class SayHelloRunnable implements Runnable {
      public static void main(String[] args) {
        (new Thread(() -> System.out.println("Hello"))).start();
      }
    }
\end{java}

\paragraph{Thread Joining}

Here: force output of ``Hello'' before ``Goodbye''
\begin{java}
  class SayHelloRunnable implements Runnable {
    public void run() {
      System.out.println("Hello");
    }
    public static void main(String[] args)
      throws InterruptedException {
      Thread thread = new Thread(new SayHelloRunnable());
      thread.start();
      thread.join();
      System.out.println("Goodbye");
    }
  }
\end{java}

\paragraph{Thread Result Passing}

Define \code{getResult}-Method returning result after joining thread
\begin{java}
  MyThread t = new MyThread();
  t.start();
  t.join();
  String result = t.getResult();
\end{java}

\paragraph{Thread Priorities}
\begin{itemize}
  \item \code{void setPriority(int priority)}: set priority of a thread
  \item by default, threads inherit priority of creating thread
  \item predefined: \code{Thread.MIN_PRIORITY}/\code{NORM_PRIORITY}/\code{MAX_PRIORITY}
\end{itemize}

\paragraph{Synchronization --- Monitors}
\begin{java}
  public void doSomething() {
    synchronized(someObject) {
      // critical, lock held on someObject
    }
  }

  // whole method critcal, monitor = this
  public synchronized void doSomething2() {
    // critical
  }
\end{java}

\paragraph{volatile Keyword}
\begin{itemize}
  \item ensures that changes to variables are immediately visible to all threads
  \item[\( \to \)] happens-before relationship: write to volatile happens-before every subsequent read of that volatile
\end{itemize}

\paragraph{Guarded Blocks}
\begin{itemize}
  \item poll condition that must be true before proceeding
  \item Idea: \code{while (!condition) \{\}; doSomething();}
  \item \textbf{Signals}: = operations that can be callled on monitors for coordination
  \item only thread ``owning'' monitor can use signals on it \( \to \) only use inside synchronized block!
  \item \code{wait()}: release monitor so another thread can enter
  \begin{itemize}
    \item thread waits for \code{notify()} or \code{notifyAll()}
    \item can throw \code{InterruptedException}
    \item overloaded method with timeout parameter available
  \end{itemize}
  \item \code{notify()}: wake up one thread that called \code{wait()} on monitor
  \begin{itemize}
    \item can wake ``wrong'' thread (e.g. in producer-consumer context)
  \end{itemize}
  \item \code{notifyAll()}: wake all threads waiting for monitor
  \begin{itemize}
    \item safer choice than \code{notify()}
  \end{itemize}
\end{itemize}
\begin{java}
  public class MyBlockingStack {
    public synchronized void push (Object o) {
      while (count == max) {
        try {
          this.wait();
        } catch (InterruptedException e) {
          // ...
        }
      }
      stack.push(o);
      this.notifyAll();
    }

    public synchronized Object pop () {
      while (count == 0) {
        try {
          this.wait();
        } catch (InterruptedException e) {
          // ...
        }
      }
      this.notifyAll();
      return stack.pop();
    }
  }
\end{java}

\paragraph{Immutable Objects}
\begin{itemize}
  \item simple way to avoid concurrency problems
  \item make class immutable:
  \begin{enumerate}
    \item declare all fields \code{private} and \code{final}
    \item no setter methods
    \item no overwriting \( \to \) declare class as \code{final}
    \item references to mutable classes are not modifiable through methods
    \item references to mutable classes are not shared
  \end{enumerate}
\end{itemize}

\paragraph{Atomic Types}

Types supporting atomic operations on single variables (e.g. AtomicInteger from java.util.concurrent.atomic)

\paragraph{Locks}
\begin{java}
  public void doSomething () {
    lock.lock();
    try {
      // critical
    } finally {
      lock.unlock();
    }
  }
\end{java}
\begin{itemize}
  \item \code{tryLock()}: acquire lock if possible (can be used to acquire several locks without blocking)
\end{itemize}

\paragraph{Counting Semaphore}
\begin{itemize}
  \item \code{Semaphore(int capacity, boolean fair)}
  \item critical section can be entered \( n \) times
  \item \textbf{permit}: one access to critical section
  \item \code{aquire()}: takes permit, reduces \# available permits, blocks if all permits aquired
  \item \code{release()}: releases permit, increases \# available permits, potentially releases blocking aquirer
  \item \code{tryAquire()}: non-blocking \code{aquire()}
  \item all calls possible with arbitrary \# permits (\code{void aquire(int amount)})
\end{itemize}

\paragraph{Barriers}
\begin{itemize}
  \item \code{CyclicBarrier(int n)}
  \begin{itemize}
    \item \code{await()} blocks calling thread --- if called \( n \) times, all threads resume
    \item can be reused afterwards
  \end{itemize}
  \item \code{CountDownLatch(int n)}
  \begin{itemize}
    \item \code{await()} blocks calling thread
    \item if \code{countdown()} called \( n \) times, all threads resume
    \item latch cannot be restarted afterwards
    \item further calls to \code{await()} return immediately
  \end{itemize}
  \code{Exchanger<V>()}: synchronous exchange of two objects between two threads
  \begin{itemize}
    \item \code{V exchange(V objectToExchange)}: exchange method has to be called by both threads, blocks until both have called
  \end{itemize}
\end{itemize}

\paragraph{Executor}
\begin{itemize}
  \item abstract from thread creation
  \item simple implementation: only start thread
  \item complex implementation: e.g. reuse already created threads
  \item \code{Executor}: simple interface, supports task execution
  \begin{itemize}
    \item \code{void execute(Runnable runnable)}
  \end{itemize}
  \item \code{ExecutorService}: most important interface
  \begin{itemize}
    \item subinterface of \code{Executor}, provides further lifecycle management logic
  \end{itemize}
  \item \code{Executor}\textcolor{red}{\footnotesize\texttt{s}}: class providing convenient factory methods for creating \code{ExecutorService}
  \begin{itemize}
    \item \code{newSingleThreadExecutor()}: creates \code{Executor} using single thread
    \item \code{newFixedThreadPool(int)}: creates thread pool with reused threads, fixed size
    \item \code{newCachedThreadPool()}: creates thread pool with reused threads, dynamic size
  \end{itemize}
\end{itemize}
\begin{java}
  ExecutorService executor = Executors.newSingleThreadExecutor();
  executor.execute(() -> {
    String threadName = Thread.currentThread().getName();
    System.out.println("Hello " + threadName);
  })

  try {
    executor.shutdown();
    executor.awaitTermination(5, TimeUnit.SECONDS);
  } catch (InterruptedException ex) {
    // handle
  } finally {
    if (!executor.isTerminated()) {
      executor.shutdownNow();
    }
  }
\end{java}

\paragraph{Callable Interface}
\begin{itemize}
  \item allows threads to return results
  \item similar to \code{Runnable}: \code{call()} instead of \code{run()}
\end{itemize}
\begin{java}
  public class MyCallable implements Callable<String> {
    int id;

    public MyCallable (int id) {
      this.id = id;
    }

    public String call () {
      return "Run " + id;
    }
  }
\end{java}

\paragraph{Futures}
\begin{java}
  ExecutorService executorService = Executors.newCachedThreadPool();
  List<Future<Integer>> futures = new ArrayList<Future<Integer>>();

  for (int i = 0; i < 10; i++) {
    final int currentValue = i;
    futures.add(executorService.submit(() -> { return currentValue; }))
  }
  for (Future<Integer> future : futures) {
    try {
      Integer result = future.get(); // like JS await
      System.out.println(result);
    } catch (ExecutorException ex) {}
  }
  executorService.shutdown();
\end{java}

\paragraph{Scheduled Executor Service}
\code{ScheduledExecutorService}: subinterface of \code{ExecutorService}, supports scheduled task execution
\begin{itemize}
  \item Future: \code{schedule(Runnable task, long delay, TimeUnit timeunit)}
  \item Periodic: \code{scheduleAtFixedRate(Runnable task, long initialDelay, long period, TimeUnit timeunit)}
\end{itemize}
\begin{java}
  ScheduledExecutorService executor = 
    Executors.newScheduledThreadPool(1);
  Runnable task = () -> System.out.println("Scheduling: "
    + System.nanoTime());
  ScheduledFuture<?> future = 
    executor.schedule(task, 3, TimeUnit.SECONDS);
  TimeUnit.MILLISECONDS.sleep(1337);
  long remainingDelay = future.getDelay(TimeUnit.MILLISECONDS);
  System.out.println("Remaining Delay: " + remainingDelay);
\end{java}

\paragraph{Completable Future}
\begin{itemize}
  \item \code{Future} drawback: caller can query result, but not register callback
  \item \code{CompletableFuture}: provides \code{thenApply}, like \code{then} in ES6
\end{itemize}
\begin{java}
  CompletableFuture<String> modified =
    futureCount.thenApply((Integer count) -> {
      int transformedValue = count * 10;
      return transformedValue;
    })
    .thenApply(transformed -> "Finally create a string: "
      + transformed);
  System.out.println(modified.get());
\end{java}

\paragraph{Fork and Join}
\begin{itemize}
  \item Java provides abstract \code{ForkJoinPool} and \code{ForkJoinTasks}
  \item Concretizations: \code{RecursiveAction} (no result), \code{RecursiveTask} (result)
  \item can be executed by \code{ForkJoinPool} calling its \code{invoke()} method
\end{itemize}
Task implementation:
\begin{java}
  public class MyTask extends RecursiveTask<Integer> {
    @Override
    public static Integer compute () {
      // calculate sth, return if problem small
      
      // divide task
      MyTask leftTask = new MyTask(...);
      MyTask rightTask = new MyTask(...);
      leftTask.fork();
      rightTask.compute(); // one subtask in-place
      leftTask.join();
  
      // compute result
    }
  }
\end{java}

\paragraph{Thread-safe Classes}
\begin{itemize}
  \item can be used safely by multiple threads concurrently
  \item \code{BlockingQueue} (interface):
  \begin{itemize}
    \item queue with ops that block if queue full/empty/putting/retrieving
    \item \code{put(...)}, \code{take()}
    \item Implementations:
    \begin{itemize}
      \item \code{ArrayBlockingQueue}: limited capacity
      \item \code{LinkedBlockingQueue}: optionally limited capacity
      \item \code{PriorityBlockingQueue}: sorted
    \end{itemize}
  \end{itemize}
  \item \code{ConcurrentHashMap}
\end{itemize}

\paragraph{Streams}
\begin{java}
  public class Person {
    private final boolean isStudent;
    private final int age;
    public boolean isStudent () { return isStudent; }
    public int getAge () { return age; }

    public Person (boolean isStudent, int age) {
      this.isStudent = isStudent;
      this.age = age;
    }
  }

  double average = personsInAuditorium
    .stream()
    .filter(Person::isStudent)
    .mapToInt(Person::getAge)
    .average()
    .getAsDouble();

  // collect op example
  // 1. supplier: supplies new result container
  // 2. accumulator: incorporates new element into result
  // 3. combiner: combines two values, must be
  //    compatible with result
  //    (note: not used here, only for parallel comp.)
  personsInAuditorium.stream().collect(
    () -> 0,
    (currentSum, person) -> { currentSum += person.getAge(); },
    (leftSum, rightSum) -> { leftSum += rightSum }
  )
\end{java}
\begin{itemize}
  \item predefined collectors exist
  \item e.g. mapping, summing up, grouping,\dots
\end{itemize}

\paragraph{Parallel Stream}
Stream executing certain operations automatically in parallel
\begin{java}
  double average = personsInAuditorium
    .parallelStream()
    .filter(Person::isStudent)
    .mapToInt(Person::getAge)
    .average()
    .getAsDouble();
\end{java}